\label{sec:discussion}
Our goal when developing VizAPI was to enable 1) library developers to make better
decisions about pruning or modifying unused APIs and to refactor their
libraries; and 2) client developers to make better decisions about library
upgrades and breaking changes.

% can we talk about the use of tests a bit more? how do we miss stuff?

Note that client tests may
not adequately represent actual client behaviours; however, our use of both static
and dynamic information addresses this issue. Specifically, because we use
class hierarchy analysis for our static analysis, our visualization will present
all possible static calls---possibly too many. 
That is, the main hazard with static analysis is that our visualization may include more
static edges than are actually possible. Some of those edges could be ruled out by a more
precise call graph. Reflection aside, no static edges
are missing (our approach is ``soundy~\cite{livshits15:_in_defen_sound}'' with respect to static information). On the other hand, dynamic edges have actually been observed
on some execution; better tests could yield more dynamic edges. But even if
a dynamic edge is missing, there will be a static edge if the behaviour is possible.

Nevertheless, this preliminary work has presented two usage scenarios which we
believe promise to be useful for both client and library
developers. We intend to carry out further user evaluations of our tool following
existing techniques~\cite{merino18:_system_liter_review_softw_visual_evaluat}; in
particular, we aspire to perform experiments to establish the
effectiveness of VizAPI, where we ask users to perform software
understanding and maintenance tasks that would benefit from our tool.

%Consistent with the Call for Papers for the NIER track, we have not yet carried out a formal evaluation of our VizAPI tool.

%Cite a systematic literature review of software visualization evaluation (Merino et al)? Merino is PC chair and talk about how we don't have an evaluation yet but it's not required for NIER.
