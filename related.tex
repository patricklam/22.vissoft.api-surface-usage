\label{sec:related-work}
A representative tool from the software visualization literature is
CodeSurveyor, by Hawes et al~\cite{hawes15:_codes}, which visualizes large
codebases using the analogy of cartographic maps. While it
incorporates dependency information into the layout of the map, VizAPI
differs from CodeSurveyor in that VizAPI focusses on usage relationships
between different modules (e.g. API invocations) using test
cases and static analysis to identify relationships between clients and libraries, rather
than investigating a particular system, as CodeSurveyor does.  Earlier
work includes the software cartography project by Kuhn et
al~\cite{kuhn10:_softw} and software terrain maps by DeLine~\cite{deline05:_stayin}.

% could mention
%software visualization/history animation:

%Gource
%code_swarm https://effectivesoftwaredesign.com/2012/05/24/code-swarm-visualizing-the-evolution-of-software-systems/

%% There is a large body of work investigating API usages, particularly
%% in mining API specifications, as first introduced by Ammons et
%% al~\cite{AmmonsETAL02MiningSpecifications}. 
%% Saied et
%% al~\cite{saied15:_minin_multi_api_usage_patter} studied which API
%% calls tended to co-occur in client code and inferred co-usage
%% relationships between these calls. Like them, we are specifically not
%% investigating sequencing relationships between API calls.

Hejderup and Gousios~\cite{DBLP:journals/jss/HejderupG22} explore a
question which is central to our approach---how well do client tests
exercise their dependencies' libraries? That is, we rely on
client test suites exercising enough of the dependencies to get valid
results from our analyses. They conclude that a combination of
static and dynamic analysis of the client has some chance of detecting
breaking changes in its dependencies, and we accordingly use static
analysis to supplement our dynamic results.

Thummalapenta and Xie~\cite{thummalapenta08:_spotw} presented the
related SpotWeb tool, which identifies framework hotspots (APIs that
are often used) and coldspots (APIs that are never used); they do not
consider framework APIs that are used but not intended to be. Hotspots
and coldspots are, however, related to our investigations about client
use of APIs; part of our study could be seen as investigating the
prevalence of hotspots on our benchmark suite. Their notion of API
usage is similar to ours, but they perform a static search to identify
uses, while we additionally dynamically observe test executions. They
also identify the top $N$ percent of used APIs as hotspots, and unused
APIs as
coldspots. Viljamaa~\cite{viljamaa03:_rever_engin_framew_reuse_inter}
also aimed to find hotspots but used concept analysis to do so.

Our overall goal is to help both client and library developers understand
client uses of library code. Clients benefit from sharpened warnings
about unsafe upgrades, knowledge that some upgrades are safe, and
having reduced attack surfaces. Library upgrades have been
investigated by many researchers, including Lam et
al~\cite{lam20:_puttin_seman_seman_version} and Kura et al~\cite{kula18:_do_devel_updat_their_librar_depen}. Kura et al found that most
software had outdated dependencies, and that software developers had
negative feelings about being required to constantly upgrade their
libraries. Being able to visualize dependencies may help developers
prioritize required upgrades as low-effort or high-effort.
Foo et al~\cite{foo18:_effic_static_check_librar_updat}
proposed a static analysis which detected safe upgrades, but could
only certify safety for 10\% of upgrades; our combined static and
dynamic approach presents the developer with more information and
enables more upgrades. 
