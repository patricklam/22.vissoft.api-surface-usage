\label{sec:vis-system}

Once we have generated data, we separate all the different interactions
into two categories---static and dynamic. We then use a modified version
of the d3graph \todo{cite} library in Python to generate a d3js \todo {cite} 
visualization. We now briefly describe the graphs generated by VizAPI.
The graphs are force-directed based on edge weights, which are the
frequency of interactions between the source and target nodes.
Each node is a set of one or more packages(or classes or methods) 
that belong to the same jar. Nodes are coalesced if they belong to the same 
jar and have the same incoming and outgoing edges. Each edge is directed 
from the source package(s) to the target package(s) and represents an interaction 
(invocations, fields, annotations, subtyping) between packages. We run a 
clustering algorithm and colour nodes based on the cluster 
(could comprise same or different jars) that they belong to. 
Hovering on a node shows you the list of packages and 
the jar they belong to, 
formatted as “jar : $<$space separated list of packages$>$”.
