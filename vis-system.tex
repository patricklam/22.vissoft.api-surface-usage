\label{sec:vis-system}

Once we have generated data, we use a modified version
of the d3graph\footnote{\url{https://pypi.org/project/d3graph/}} library in Python to generate a d3js\footnote{\url{https://d3js.org/}}
visualization. 

VizAPI graphs are force-directed graphs based on edge weights, which we assign
according to the
frequency of interactions between the source and target nodes.
Each node is a set of one or more packages (or classes or methods) 
that belong to the same JAR. We coalesce nodes if they originate from the same 
JAR file and have the same incoming and outgoing edges. Each edge is directed 
from the source package(s) to the target package(s) and represents an interaction 
(invocations, fields, annotations, subtyping) between packages. We run a 
clustering algorithm and use its output to colour nodes based on the cluster 
that they belong to, so a colour could include nodes from the same or different JARs.
Hovering on a node shows the list of packages and 
the JAR that they belong to, 
formatted as “jar : $<$space separated list of packages$>$”.
